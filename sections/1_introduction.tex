%!TEX root = ../thesis.tex
\chapter{Introduction}
In data mining and statistics deriving knowledge from observational data is used to find correlations between variables in the data. Through those correlations data distribution characteristics are found

% motivation, Relevanz (Anwendungsfall, Daten etc warum Effizienz?)

Moore's Law \cite{mooreCrammingMoreComponents1965} says essentially that transistors placed on an integrated circuit double approximately every two years. This law has been self-fulfilling and can be seen in the real world. Still modern processor development is getting to a physical limit, called the "Power Wall". More transistors, that are more densely placed, with lower voltage, and higher frequencies, produce all together more heat. While cooling performance has its physical limits, the "Power Wall" is hit at some point. To work against this limit, modern processors incorporate multiple cores, that work in parallel.

Multi-core processors can speedup execution, by splitting work and processing that work in parallel. But parallelization has its own limits and converges to an upper bound  \cite{amdahlValiditySingleProcessor1967}.


% Trend hardware? heterogene (vllt dark silicon)
% moores law
% processing speed i limited an does not grow any longer (physics, get too hot when more dense) => dynamic power, power wall
% ILP wall? maybe too detailed
% memory wall (memory speed, and latency limited)
% Dark silicon = next step / power wall 2.0
% - multicore scaling limited as well
\cite{esmaeilzadehDarkSiliconEnd2011}
% - specialized hardware has potential
% - speedup through using the right processor for the correct task
% - Needs alot more software optimizations
% speedup for what?
% - medical causations => The faster, the better
% - when doing multiple experiments/iterations on data, small speedup can mean alot
% - energy efficiency, by optimal resource usage
% - some processing units are more efficient

% Outline, structure

% Teaser auf problem statement

% Womit muss man sich auseinandersetzen (load balanacing)