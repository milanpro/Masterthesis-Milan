%!TEX root = ../thesis.tex
\chapter*{Zusammenfassung}
\hyphenation{kau-sa-le}
Constraint-basierte kausale Strukturlernalgorithmen werden verwendet, um kausale Beziehungen zwischen Variablen aus Beobachtungsdaten abzuleiten. Methoden nach dem Stand der Technik sind vor allem durch ihre lange Ausführungszeit begrenzt, insbesondere bei großen und hochdimensionalen Datensätzen. Die Leistungsfähigkeit von Constraint basierten kausalen Strukturlernalgorithmen wie dem PC-Algorithmus ist essentiell für zeitabhängiges und energieeffizientes Rechnen. Moderne High-Performance-Computing-Systeme bestehen hauptsächlich aus konventionellen Prozessoren, die von spezialisierten Recheneinheiten wie der Graphics Processing Unit (GPU) unterstützt werden. Algorithmen, die mit Blick auf heterogene Systeme entwickelt wurden, nutzen alle Systemressourcen und streben eine optimale Prozessorauslastung an. Der PC-Algorithmus, der nicht für heterogene Systeme konzipiert ist, muss an das heterogene Computing angepasst werden. In dieser Arbeit wird der PC-Algorithmus mit zwei verschiedenen Ansätzen parallelisiert, um heterogene CPU-GPU Systeme zu nutzen. Der PC-Algorithmus ist unabhängig von der zugrunde liegenden Datensatzverteilung definiert. In dieser Arbeit wird jedoch davon ausgegangen, dass die Datensätze multivariat normalverteilt sind. Der erste Ansatz ist ein statischer Scheduling-Ansatz, der die Arbeit vor der Ausführung verteilt, und der zweite Ansatz basiert auf Workstealing-Algorithmen, die die aufgeteilten Arbeitspakete während der Ausführung dynamisch verteilen. Eine experimentelle Auswertung beider Ansätze zeigt, dass eine Beschleunigung erreichbar ist, auch wenn Kommunikations- und Balancierungs-Overhead den Parallelisierungseffekt begrenzen können. Die Evaluierung der Ansätze im Vergleich zum rein GPU-beschleunigten PC-Algorithmus zeigt, dass das dynamische CPU/GPU Scheduling durch den Workstealing-basierten Ansatz die reine GPU-Lösung übertrifft mit einer um 16,5\% besseren Laufzeit. In Situationen, in denen Speichergrenzen der GPU erreicht werden, sind Beschleunigungsfaktoren von bis zu 6,61 möglich.