%!TEX root = ../thesis.tex
\chapter{Related Work}
In the context of Constaint Based Causal Structure Learning or more specifically the PC algorithm, there is related work with a focus on accelerating and improving the performance through utilization of the hardware. The most indepth researched question is based on the general performance on CPU based systems \cite{leFastPCAlgorithm2019, leParallelPCPackageEfficient2015, schmidtLoadBalancedParallelConstraintBased2019, colomboOrderIndependentConstraintBasedCausal,kalischEstimatingHighDimensionalDirected2007,scutariBayesianNetworkConstraintBased2017}. Based on the work of the CPU based PC algorithm approaches, related work for a few GPU accelerated variants \cite{schmidtOrderIndependentConstraintBasedCausal2018,zarebavaniCuPCCUDAbasedParallel2018} and load-balancing \cite{schmidtLoadBalancedParallelConstraintBased2019} exists.

The related work for the PC algorithm is highly relevant for the work done in this thesis. The base implementation is heavily influenced by them and helped forming the heterogeneous computing variant.

For an entry into the heterogeneous computing topic surveys such as heterogeneous computing techniques by Mittal and Vetter \cite{mittalSurveyCPUGPUHeterogeneous2015} or an introduction into heterogeneous systems by Carabano et al. \cite{carabanoExplorationHeterogeneousSystems2013}, where differences between the CPU and GPU are elaborated and clarified. Such work supports the argumentation of both the GPUs and CPUs properties to look at. With that a better scheduling can be achieved. By looking into the work of Zhang \cite{zhangDynamicStaticLoad1991} the differences between static and dynamic scheduling are explained, which formed the two different approaches based on those scheduling variants.

In every heterogeneous computing, tasks have to be scheduled on different processing units, therefore related work based on this is also relavant for this thesis. There is work for the genereal view on the heterogeneous computung scheduling and load balancing topic \cite{abdelkaderDynamicTaskScheduling2012,binottoDynamicReconfigurableLoadbalancing2010,galindoDynamicLoadBalancing2008,kopetzRealTimeScheduling1997,kwokStaticSchedulingAlgorithms1999,momcilovicDynamicLoadBalancing2014,singhSurveyStaticScheduling2015}. By inspecting the work done on load balancing, the workstealing scheduler seemed best fitting for the PC algorithm use-case, next to the basic static scheduling (pre-balanced) approach.

Workstealing as a concept evolved from being used in CPU thread scheduling \cite{blumofeSchedulingMultithreadedComputations1999} into being relevant in parallel computing as well \cite{letzWorkStealingScheduler2010,mattheisWorkStealingStrategies2012,prellEmbracingExplicitCommunication2016}. The idea of the workstealing approach is based on this work.

