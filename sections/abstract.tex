%!TEX root = ../thesis.tex
\chapter*{Abstract}
Constraint-based causal structure learning algorithms are used to derive causal relationships between variables in observational data. State-of-the-art methods are limited mainly by their long execution time, especially for large and high-dimensional datasets. The performance of constraint-based causal structure learning algorithms such as the PC algorithm is essential for time-dependent and energy-efficient computing. Modern high-performance computing systems mainly consist of conventional processors assisted by specialized processing units such as the graphics processing unit (GPU). Algorithms developed with heterogeneous systems in mind use the system resources and enable full processor utilization. The PC algorithm, which is not designed for heterogeneous systems, has to be adapted to heterogeneous computing. In this thesis, the PC algorithm is parallelized using two different approaches to utilize heterogeneous CPU-GPU systems. The PC algorithm is defined independently from the underlying dataset distribution. Yet, in this thesis datasets are assumed to be multivariate normal distributed. The first approach is a static scheduling approach that balances the work pre-execution, and the second approach is based on workstealing algorithms balancing the split work during execution dynamically. An experimental evaluation of both approaches shows that speedup is achievable even if communication and balancing overhead can limit the parallelization effect. Evaluating the approaches in comparison to the GPU-only accelerated PC algorithm shows that dynamic scheduling on both the CPU and GPU through the workstealing-based approach outperforms the GPU-only solution with a 16,5\% better execution time. In situations where memory limits are reached on the GPU, speedup factors of up to 6,61 are possible.
% Outline the whole thesis 
% motivation etc
% what steps are done
% which results where found