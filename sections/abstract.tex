%!TEX root = ../thesis.tex
\chapter*{Abstract}
Constraint-based causal structure learning algorithms are used to gather causal relationships between variables in data. State-of-the-art methods are mostly limited by their long execution time, especially fro large and high dimensional datasets. The performance of constraint-based causal structure learning algorithms such as the PC algorithm are therefore essential for time dependent results and energy efficient computing. Modern high-performance computing systems mostly consist of specialized processing units such as the CPU and GPU. Heterogeneous computing algorithms use those resources and enable full utilization. Algorithms such as the PC algorithm have to be adapted to heterogeneous computing. In this thesis, tthe PC algorithm is parallelized using two different approaches to utilize heterogeneous CPU-GPU systems and it is shown, that a speedup is achievable even if communication overhead can limit this effect. Evaluating the approaches in comparison to the GPU-only accelerated PC-algorithm show, that dynamic scheduling on both the CPU and GPU through a workstealing based approach outperforms the GPU-only solution by 16,5\%. In situations where memory limits are reached on the GPU, speedup factors of 6,61x are possible.
% Outline the whole thesis
% motivation etc
% what steps are done
% which results where found

\chapter*{Zusammenfassung}
% abstract übersetzen
Constraint-basierte Kausalstruktur-Lernalgorithmen werden verwendet, um kausale Beziehungen zwischen Variablen in Daten zu erfassen. Methoden nach dem Stand der Technik sind meist durch ihre lange Ausführungszeit begrenzt, insbesondere bei großen und hochdimensionalen Datensätzen. Die Leistungsfähigkeit von Constraint-basierten Kausalstruktur-Lernalgorithmen wie dem PC-Algorithmus sind daher essentiell für zeitabhängige Ergebnisse und energieeffizientes Rechnen. Moderne High-Performance-Computing-Systeme bestehen meist aus spezialisierten Recheneinheiten wie der CPU und GPU. Heterogene Rechenalgorithmen nutzen diese Ressourcen und ermöglichen die volle Ausnutzung. Algorithmen wie der PC-Algorithmus müssen an heterogenes Rechnen angepasst werden. In dieser Arbeit wird der PC-Algorithmus mit zwei verschiedenen Ansätzen zur Ausnutzung heterogener CPU-GPU-Systeme parallelisiert und es wird gezeigt, dass ein Speedup erreichbar ist, auch wenn der Kommunikationsoverhead diesen Effekt einschränken kann. Die Auswertung der Ansätze im Vergleich zum GPU-only beschleunigten PC-Algorithmus zeigt, dass das dynamische Scheduling sowohl auf der CPU als auch auf der GPU durch einen Workstealing-basierten Ansatz die GPU-only-Lösung um 16,5\% übertrifft. In Situationen, in denen Speichergrenzen auf der GPU erreicht werden, sind Beschleunigungsfaktoren von 6,61x möglich.
