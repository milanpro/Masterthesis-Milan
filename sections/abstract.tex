%!TEX root = ../thesis.tex
\chapter*{Abstract}
Constraint-based causal structure learning algorithms are used to derive causal relationships between variables in observational data. State-of-the-art methods are limited mainly by their long execution time, especially for large and high-dimensional datasets. The performance of constraint-based causal structure learning algorithms such as the PC algorithm is essential for time-dependent and energy-efficient computing. Modern high-performance computing systems mainly consist of conventional processors assisted by specialized processing units such as the graphics processing unit. Algorithms developed with heterogeneous systems in mind use the system resources and enable full processor utilization. The PC algorithm, which is not designed for heterogeneous systems, has to be adapted to heterogeneous computing. In this thesis, the PC algorithm is parallelized using two different approaches to utilize heterogeneous CPU-GPU systems. The PC algorithm is defined independently from the underlying dataset distribution. In this thesis the focus lies on multivariate distributed datasets. The first approach is a static scheduling approach that balances the work pre-execution, and the second approach is based on workstealing algorithms balancing the split work during execution dynamically. An experimental evaluation of both approaches shows that speedup is achievable even if communication and balancing overhead can limit the parallelization effect. Evaluating the approaches in comparison to the GPU-only accelerated PC algorithm shows that dynamic scheduling on both the CPU and GPU through the workstealing based approach outperforms the GPU-only solution by 16,5\%. In situations where memory limits are reached on the GPU, speedup factors of up to 6,61 are possible.
% Outline the whole thesis
% motivation etc
% what steps are done
% which results where found

\chapter*{Zusammenfassung}
% abstract übersetzen
Constraint-basierte Kausalstruktur-Lernalgorithmen werden verwendet, um kausale Beziehungen zwischen Variablen in Daten zu erfassen. Methoden nach dem Stand der Technik sind vor allem durch ihre lange Ausführungszeit begrenzt, insbesondere bei großen und hochdimensionalen Datensätzen. Die Leistungsfähigkeit von Constraint-basierten Kausalstruktur-Lernalgorithmen wie dem PC-Algorithmus ist essentiell für zeitabhängiges und energieeffizientes Rechnen. Moderne Hochleistungscomputersysteme bestehen hauptsächlich aus spezialisierten Prozessoreinheiten wie der CPU und GPU. Heterogene Rechenalgorithmen nutzen diese Ressourcen und ermöglichen die volle Ausnutzung. Algorithmen wie der PC-Algorithmus müssen an heterogenes Rechnen angepasst werden. In dieser Arbeit wird der PC-Algorithmus mit zwei verschiedenen Ansätzen parallelisiert, um heterogene CPU-GPU-Systeme zu nutzen. Es wird gezeigt, dass ein Beschleunigungseffekt erzielt werden kann, auch wenn der Kommunikationsaufwand den Parallelisierungseffekt begrenzt. Die Auswertung der Ansätze im Vergleich zum rein GPU-beschleunigten PC-Algorithmus zeigt, dass das dynamische Scheduling sowohl auf der CPU als auch auf der GPU durch einen Workstealing-basierten Ansatz die reine GPU-Lösung um 16,5\% übertrifft. In Situationen, in denen Speichergrenzen auf der GPU erreicht werden, sind Beschleunigungsfaktoren von 6,61x möglich.
