%!TEX root = ../thesis.tex
\chapter{Problem Statement}
Efforts for a more efficient PC Algorithm have been made for versions running solely on the CPU or GPU. Since the skeleton discovery phase can be highly parallelizable, GPU variants mostly outperformed variants running on the CPU. In both variants, the other processor idles, and its processing power is unused. A solution to this problem could be using both processing units simultaneously and split the workload accordingly.
Splitting the workload into predefined tasks can help to parallelize the workload. For scheduling such split tasks on homogeneous processing units, a simple load balancing such as splitting the task load into equal size batches and placing them on the available processors is sufficient.
Because of their equal processing power, those homogeneous processors should process these task batches in the same time frame.
Heterogeneous systems are more complex than homogeneous for balancing the workload because of the additional parameters influencing the execution. For example, a modern CPU has few but fast processing cores, being good at processing a few tasks with high load. In contrast, the modern GPU has multiple slower processing cores that process many parallel tasks concurrently.

On this basis, the following research questions arise: How to optimally use the resources of a heterogeneous system to speed up the PC Algorithm execution?
The research question splits into a few smaller questions that are important to answer the research question itself and allow a finer-grained view of the problem:

\begin{enumerate}
  \item What tasks to define in the workload of the PC-Algorithm?
  \item How to parallelize the workload in a heterogeneous context?
  \item How to schedule the workload effectively on those processing units?
  \item What performance parameters are important on heterogeneous systems?
\end{enumerate}

Deciding on a task size is the main factor in being able to schedule the workload effectively. Using this task to parallelize the workload is then effectively assigning the tasks to the given processing units. Assigning the task to the given processing units has to be done by some scheduling algorithm, which has to know about the performance parameters of the heterogeneous system and its processing units.
