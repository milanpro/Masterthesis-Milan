%!TEX root = ../thesis.tex
\chapter{Future Work}
\label{chap:fuwork}
This thesis experiments rely on the same dataset with different sized variable samples. Although this helps to get a first good impression on the effectiveness of the approaches, there is still a bias in the experiment result regarding the datasets structure and edge deletion behavior. If the edge deletion state of the graph is different, there is expected to be a different effectiveness outcome. While a more significant difference in outlier versus non-outlier task length could result in more effective outlier scheduling on the CPU, less to no outlier occurrence should make the GPU-only variant more effective.

This dataset behavior could be tested by executing the benchmarks with a different alpha value, directly influencing the amount with which edges are deleted. Another way of testing is generating datasets that show one of the said behaviors.

For better comparison, other real-world dataset experiments could be done. Multiple gene expression datasets are often used in the context of benchmarking the PC algorithm, such as NCI-60, MCC, or BR51 \cite{leFastPCAlgorithm2019}. By executing benchmarks with said datasets, direct comparisons to state-of-the-art methods could be made to understand the performance implications of both approaches better. Additionally insights about the appraoches dependency on the dataset and the effectivness in other real-world cases can be derived.

The approaches shown in this thesis are not dependent on some specific CI-test. For simplicity, every benchmark was done using a CI-test that fits the multivariate normal distribution. For other distributions such as categorical data, other tests are used \cite{scutariLearningBayesianNetworks2010}. Those other tests require other data structures and can change the data access patterns significantly. With other data access patterns, the data dependencies between tasks change, and therefore, the interconnect usage between CPU and GPU has to be considered again algorithm design wise. Testing other CI-test could help to get insights on how those tests fit the heterogeneous computing setting.

Real world datasets are not always composed of variables distributed in the same way. Mixed distribution datasets are also called hetergoeneous datasets and require more complex conditional independence tests. Different CI-tests are possibly better executed on different specialized processing units. In case of hetergoeneous datasets a more complex load-balancing could be introduced, that balances tasks also based on the CI-test used in this task.

The two used heterogeneous systems called Delos and AC922 the experiments are based on show that the underlying system/hardware specifications are essential for the execution of heterogeneous computing. Other systems could be added to the experiments to understand the influences of the underlying hardware. Those systems could feature different performance disparities between CPU and GPU to get further insights into the effects of the hardware in contrast to the interconnect. Adding multiple GPU benchmarks could also be interesting to test and profile execution times. Sometimes two different system GPUs are directly connected to two different system CPUs. In such a system, a more intelligent scheduling algorithm could be beneficial, identifying which task to steal from which GPU for best performance.

Heterogeneous systems do not only consist of CPUs and GPUs. Many different kinds of processing units could also support the algorithm execution. One of the also widely know processing units is called Field Programmable Gate Array (FPGA). FPGAs can be programmed to mimic a computer chip specifically designed for a purpose. The developer is in charge of deciding what gate is connected to which and designs hardware via software. An FPGA is more efficient in comparison to CPU and GPU when used correctly, \cite{qasaimehComparingEnergyEfficiency2019} and could be added to the heterogeneous computing solution proposed in this thesis to accelerate further.
% Test with different dataset charateristics
% Test with Stronger CPU/GPU disparity
% Add FPGA
% 